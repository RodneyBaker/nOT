\documentclass[a4paper,12pt]{article}
\usepackage[a4paper, total={180mm, 272mm}]{geometry}

\usepackage{fontspec}
\setmainfont[Path=fonts/, Extension=.ttf]{ipaexm}

\setlength\parindent{3.5em}
\setlength\parskip{0em}
\renewcommand{\baselinestretch}{1.247}

\begin{document}

\thispagestyle{empty}

\Large
\noindent \\
Motion Blur Ino\medskip
\par
\normalsize
平行移動によるカメラシャッター風のぼかしを行ないます。\par
オプション指定によりぶれ残像効果も可能です。\\
\par
初めに、(ON 指定ならば)Alpha チャンネルに対して処理し、次に、\par
Alpha チャンネルがゼロでないピクセルの RGB を処理します。\\
\par
Alpha チャンネルに対して処理しない(OFF)指定をしたときは、\par
RGB 画像の変化をマスクします。よって、滑らかなエッジは滑らか\par
なままです。\\
\\
-{-}- \ 入力 \ -{-}-\\
Source\par
処理をする画像を接続します。\\
\\
-{-}- \ 設定 \ -{-}-\\
Depend Move\par
\noindent \ \ \, P1 -> P2\par
以下の X1,Y1,X2,Y2により、\par
固定した方向と大きさで移動を指定します。\\
\par
\noindent \ \ \, Motion\par
Geometry の、X,Y からフレーム毎に移動を指定します。\par
X1,Y1,X2,Y2の指定値は無視します。\\
\\
X1\\
Y1\\
X2\\
Y2\par
平行移動ぼかしの始点座標値と終点座標値を指定します。\par
座標系は左下が原点です。\par
単位はミリメートルです。\par
小数点以下の指定により、微妙な長短の変化がつきます。\par
始点と終点の距離が処理上1/16Pixel 以上ないと効果はでません。\par
初期値は\par
\noindent \hskip 7em X1 Y1 -> 0.0 0.0\par
\noindent \hskip 7em X2 Y2 -> 1.0 1.0\par
です。

\newpage

\thispagestyle{empty}

\ \vspace{-0.2em}
\par
\noindent Scale\par
平行移動ぼかしの長さに対してスケール調整をします。\par
例えば、\par
\noindent \hskip 7em X1 Y1 -> 0.0 0.0\par
\noindent \hskip 7em X2 Y2 -> 1.0 -1.0\par
\noindent \hskip 7em Scale -> 100\par
は、\par
\noindent \hskip 7em X1 Y1 -> 0.0 0.0\par
\noindent \hskip 7em X2 Y2 -> 100.0 -100.0\par
と同じ効果となります。\par
ゼロを指定するとぼかしはかからなくなります。\par
初期値は1でスケールはかかりません。\\
\\
Curve\par
ぼかしの強さに対して調整をします。\par
10.0以下、1より大きくすると、ぼかしが強くなり、\par
0.1以上、1より小さくすると、ぼかしは薄くなります。\par
初期値は1で等減衰です。\\
\\
Zanzo Length\par
残像効果のずれ位置を指定します\par
単位はミリメートルです。\par
ゼロ以上の値で指定します。\par
例えば、幅が3の線の残像を出したいときは、\par
3以上の値を指定しないと残像はでません。\par
初期値は0で残像はでません\\
\\
Zanzo Power\par
残像をだすときの強さを決めます。\par
最弱が0で、残像効果はかかりません。\par
値が大きくなるほど、\par
ぼかしは薄まり、残像効果が強くなります。\par
初期値は1で最強です。ぼかしはかからず残像効果のみとなります。\\
\\
Alpha Rendering\par
Alpha チャンネルがあるときのみ有効なスイッチです。\par
OFF のときは、RGB 値の変化を Alpha 値でマスクします。\par
ON で Alpha にも処理をします。マスクはしません。\par
初期値は ON です。

\end{document}