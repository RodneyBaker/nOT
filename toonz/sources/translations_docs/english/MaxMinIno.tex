\documentclass[a4paper,12pt]{article}
\usepackage[a4paper, total={180mm, 272mm}]{geometry}

\usepackage{fontspec}
\setmainfont[Path=fonts/, Extension=.ttf]{ipaexm}

\setlength\parindent{3.5em}
\setlength\parskip{0em}
\renewcommand{\baselinestretch}{1.247}

\begin{document}

\thispagestyle{empty}

\Large
\noindent \\
Max Min Ino\medskip
\par
\normalsize
It inflates the brightest (darkest) portions of the image.\\
\par
The shape can be round or polygonal.\par
It produces smooth changes.\\
\par
First, it processes the Alpha channel, if specified.\par
Then it handles RGB values of pixels, where the Alpha channel is not zero.\\
\\
-{-}- \ Inputs \ -{-}-\\
Source\par
Connect the image to be processed.\\
\\
Reference\par
Connect the reference image to assign the strength of the effect into each pixel.\\
\\
-{-}- \ Settings \ -{-}-\\
Max Min Select\par
Specify the processing method.\par
\textquotedbl Max\textquotedbl \ -> Inflate the bright areas of the image\par
\textquotedbl Min\textquotedbl \ -> Inflate the dark areas of the image\par
When using \textquotedbl Min\textquotedbl , black lines in the cell image outline will be painted with 0\par 
in the transparent area outside them, so transparent area expands and black lines\par 
dissappear.\par
It also increases the region defined by the Alpha in the same way.\par
The default setting is \textquotedbl Max\textquotedbl .\\
\\
Radius\par
Specify the size of the bulge by a circle radius.\par
The unit is millimeters.\par
Specify a number greater than or equal to 0.\par
By adding smoothing (in pixels) it will not inflate with values smaller than 1.\par
Therefore, if the value is less, there will be an effect with a fine image,\par
but it may not make the effect noticeable on a rough image.\par
A larger Radius will take more time to process.\\

\newpage

\thispagestyle{empty}

\ \vspace{-0.2em}
\\
Polygon Number\par
Allows to specify if inflate into a circle or a polygon shape.\par
Specify an integer value.\par
A value of 2 will inflate using a circle of the specified Radius.\par
3 or more, will inflate to the number of sides of the polygon. The maximum is 16.\par
Polygons begin from the right side of the center of the bulge.\par
The default value is 2.
\\
\par
\noindent Degree\par
Specifies the angle of the polygon, when \textquotedbl Polygon Number\textquotedbl \ value is 3 or more.\par
When \textquotedbl Polygon Number\textquotedbl \ value is 2, it will have no effect.\par
Specify a value of 0 or more degrees.\par
It will rotate in clockwise direction.\par
The default value is 0.\\
\\
Alpha Rendering\par
This option is valid only when there is an Alpha channel.\par
When inactive, it masks the changes in the RGB values using the original Alpha\par 
of the image.\par
When active, the effect will be able to modify the Alpha channel, extending it\par 
as necessary to reproduce the full span of the effect.\par
The default setting is ON.\\
\\
Reference\par
Specify which channel to use from the image connected to the Reference port to\par 
drive the intensity of the effect.\par
Choose from Red/Green/Blue/Alpha/Luminance.\par
Choose Nothing to disable the effect.\par
The default value is \textquotedbl Red\textquotedbl .

\end{document}